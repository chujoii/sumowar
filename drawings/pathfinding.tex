/*
  Возьмём старую идею которая была предложена мной в проекте
  расшифровки спектров полученных фотографированием интерферометра
  Фабри-Перо. Предположим что изображение содержит окружность
  (например чёрные точки - фон, белые - объекты в данном случае
  окружности). Через три точки на плоскости можно построить только
  одну окружность. Взяв любые три белые точки из изображения и находим
  центр (xi, yi) окружности которая бы проходила через эти три
  точки. Изображение состоит из множества белых точек, поэтому для
  большей точности можно (необязательно) повторить поиск центра для
  всех возможных комбинаций белых точек из изображения. Некоторая
  функция (среднее, медиана, ...) от полученного набора координат и
  будет наиболее приближена к реальному центру окружности.

  Упрощённый метод: Предположим что камера (из оптического сенсора
  мыши) вместе с линзой охватывают только область под роботом. Линии
  имеют радиус закругления больше чем размер область обзора камеры.
  Чем дальше находятся точки друг от друга тем больше точность (если
  взять координаты соседних белых пикселей то они скорее всего дадут
  неверные координаты центра круга). Возьмём координаты точек входа
  "a" и выхода "b" окружности из кадра (окружность больше обзора
  камеры и не поместилась в кадр целиком). проведём линию между "a" и
  "b" и из центра опустим перпендикуляр. Ближайшую к перпендикуляру
  белую точку назовём "c". Комбинация точек (a, b, c) позволит найти
  центр окружности.

  Сложности:

  1) точек входа (выхода) может быть несколько "a" и рядом с ней ещё
  одна "a" (линия отображается линзой на несколько пикселей) - это не
  страшно, ведь линия широкая и робот не выйдет за границы, но
  возможно он выберет не оптимальный путь (ошибка ~2cm - ширина линии)
  и проедет не по внутреннему а по наружному краю при повороте, что
  добавит дополнительные секунды к общему времени.

  2) точки типа "d" на краю изображения затруднят поиск точек входа и
  выхода окружности.
  
  решение: обрезаем края изображения и повторяем поиск точек входа и
  выхода

  3) аналогично, артефакты типа "e" которые окажутся близко к
  перпендикуляру.
  
  плохое решение: запоминаем предыдущие координаты центра. и если
  новые координаты центра будут очень далеко то это точка "e", а не
  точка "c". ищем новую точку. а плохое оно потому, что если линия
  представляет собой букву "S" то при переходе от нижнего полукруга к
  верхнему центр окружности перескочит на большое расстояние.

  хорошее решение?

  4) места пересечения линии с собой "f".

  ?

  5) линия не круг. в некоторых местах она почти прямая - возможно
  переполнение при вычислениях (радиус прямой - бесконечен)

*/
